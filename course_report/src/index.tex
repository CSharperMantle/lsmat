\renewcommand\contentsname{\textcolor{xblue}{目录}}
    \tableofcontents
\clearpage
\setcounter{page}{1}

\section{问题描述}

稀疏矩阵 (sparse matrix), 在数值分析中, 是其元素大部分为零的矩阵. 反之, 如果大部分元素都非零, 则这个矩阵是稠密 (dense) 的. 在科学与工程领域中求解线性模型时经常出现大型的稀疏矩阵.

在使用计算机存储和操作稀疏矩阵时, 经常需要修改标准算法以利用矩阵的稀疏结构. 由于其自身的稀疏特性, 通过压缩可以大大节省稀疏矩阵的内存代价. 更为重要的是, 由于过大的尺寸, 标准的算法经常无法操作这些稀疏矩阵.

如 \ref{eq_1_smat_example} 所示为一个 $5 \times 5$ 的稀疏矩阵, 其中只有四个元素为非零值. 为了清楚起见, 零用点表示.
\begin{equation}
    \begin{bmatrix}
        . & 1 & . & . & . \\
        . & . & . & . & . \\
        . & . & . & 2 & . \\
        . & 3 & . & . & . \\
        . & . & . & 4 & . \\
    \end{bmatrix}.
    \label{eq_1_smat_example}
\end{equation}

稀疏矩阵的存储可以使用十字链表等方式储存. 可以将矩阵中的非零元表示为链表中的节点, 指向其同行和同列之后的节点, 就组成了十字链表, 便于元素的访问和增删等操作. 每个链表节点除了储存自身的行列位置信息和值外, 还储存了同行列下一个非零节点的指针信息, 构成了二维形式的链表.

在本次实践作业中, 需要基于十字链表实现对稀疏矩阵的创建, 储存和加法, 减法, 乘法, 转置等运算操作, 也可以自行添加新的数据导入导出, 矩阵运算等操作. 其中至少需要实现稀疏矩阵的创建和输入输出, 加法, 减法, 乘法和转置操作, 并给出相应算法的复杂度分析. 对于运算操作中可能出现的错误如尺寸不匹配, 不可逆 (若实现了求逆操作) 等情况, 需要能够捕获错误并提示. 在代码实现过程中还需要避免可能发生的内存泄漏.

\section{主要算法}

\subsection{数据结构描述}

一个完整的十字链表可以表示为三元组 $M = \left\langle \bm{r}, \bm{c}, N\right\rangle,$ 其中各分量分别表示行首数组、 列首数组和数据节点集合. 对于矩阵中的所有非零元, 存在 $n \in N,$ 且每个 $n = \left\langle v, \bm{a} \right\rangle,$ 其中 $\bm{a}$ 为各维度上的坐标与邻接信息数组, 每个 $\bm{a}$ 的分量都可以表示为三元组 $\bm{a}[j] = \left\langle i \in \mathbb{Z}_+, prev \in N, next \in N\right\rangle,$ 分别为该元素在该维度中的下标、 前驱和后继. 对于每个存在非零元的行或列, 其行、 列首数组的相应位置存在指向该行或列的首个非零元素的一条边.

\subsection{矩阵建立与销毁}

算法 \ref{algo_smat_init} 给出了创建一个具有给定形状 $\bm{s} \in \mathbb{R}^2$ 且元素全为 0 的系数矩阵的过程. 不难发现, 基于十字链表的稀疏矩阵在创建时并不需要为任何元素分配实际的存储空间. 实际实现中, 创建时的内存开销主要集中于为 $\bm{r}$ 和 $\bm{c}$ 所分配的空间上.

\begin{breakablealgorithm}
\caption{创建稀疏矩阵.}
\label{algo_smat_init}
\begin{algorithmic}[1]
\Require 矩阵形状 $\bm{s} \in \mathbb{R}^2$.
\Ensure 指定形状的全零矩阵.
\Function{SMatrixInit}{$\bm{s}$}
    \State $\bm{r} \gets$ 长度为 $\bm{s}[0]$ 的空数组 
    \State $\bm{c} \gets$ 长度为 $\bm{s}[0]$ 的空数组
    \State $N \gets \emptyset$
    \State \Return $\left\langle \bm{r}, \bm{c}, N\right\rangle$
\EndFunction
\end{algorithmic}
\end{breakablealgorithm}

算法 \ref{algo_maze_explore} 描述了利用回溯法实现的迷宫探索. 同样, 我们记 $\mathsf{Push}(\cdot, \cdot)$ 与 $\mathsf{Pop}(\cdot)$ 为压入与弹出操作, $\mathsf{Empty}(\cdot)$ 为判断栈是否为空的操作, $\mathsf{Top}(\cdot)$ 为获取栈顶元素的操作.

\begin{breakablealgorithm}
\caption{回溯法迷宫探索.}
\label{algo_maze_explore}
\begin{algorithmic}[1]
\Require 迷宫 $M.$
\Ensure 是否能够找到一条可行路线.
\State 初始化栈 $s$
\State $P \gets \mathrm{NULL}$
\State $\mathsf{Push}(s, \text{初始坐标})$
\While{$\lnot \mathsf{Empty}(s)$}
    \State $P \gets \mathsf{Top}(s)$
    \State 标记 $P$ 为已探索过的道路
    \For{$P' \gets \{ P\ \text{的右、 左、 上、 下方坐标}\}$}
        \If{$P'$ 在迷宫内部, 且为未探索过的道路}
            \State $\mathsf{Push}(s, P')$
            \State 跳过剩余检测, 继续 \textbf{while} 循环的下一迭代
        \EndIf
    \EndFor
    \State $\mathsf{Pop}(s)$
\EndWhile
\If{$P$ 为终点}
    \State \Return 能够找到一条可行路线
\Else
    \State \Return 无法找到一条可行路线
\EndIf
\end{algorithmic}
\end{breakablealgorithm}

\subsection{矩阵元素的增、 删、 查、 改}

\subsection{顺序访问优化}

\subsection{基础矩阵算术}

\subsection{矩阵转置}

\section{实验设计}

本实验的源代码获取方式见附录 \ref{appendix_source_code}.

\section{结果分析}

TODO.

\section{小结}

本文实现了迭代快速排序和回溯法迷宫探索两种算法, 并以可交互的形式对后者结果进行展示.  实验中, 作者对递归与试探回溯的原理有了更深刻的理解, 对如何便捷地通过控制台与用户进行图形化交互也有了更深入的了解.

\begin{thebibliography}{}

\bibitem{bib_ncurses} NCURSES - New Curses \textbf{[OL]}. https://invisible-island.net/ncurses/
\bibitem{bib_pdcurses} PDCurses - a curses library for environments that don't fit the termcap/terminfo model. \textbf{[OL]}. https://pdcurses.org/

\end{thebibliography}

\appendix

\section{附录: 算法实现}

本节给出本文实现的算法的源代码.

\subsection{稀疏矩阵基础设施}

\lstinputlisting[language=C]{./../../include/lsmat/lsmat.h}

\lstinputlisting[language=C]{./../../src/lsmat/lsmat.c}

\subsection{矩阵运算}

\lstinputlisting[language=C]{./../../include/lsmat/lsarith.h}

\lstinputlisting[language=C]{./../../src/lsmat/lsarith.c}

\subsection{用户交互界面}

\lstinputlisting[language=C]{./../../src/cli/main.c}

\section{附录: 源代码可见性}
\label{appendix_source_code}

本项目代码可于 \url{https://github.com/CSharperMantle/lsmat} 获取, 并在 BSD 3-Clause 许可证的条件与条款下使用. 授权协议的具体信息见上述 URL.
